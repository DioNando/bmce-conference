\documentclass{beamer}

% Setting the theme and font
\usetheme{Madrid}
\usecolortheme{default}
\usepackage{noto}
\usepackage[T1]{fontenc}
\usepackage[utf8]{inputenc}
\usepackage[french]{babel}
\usepackage{graphicx}
\usepackage{amsmath}
\usepackage{booktabs}
\usepackage{xcolor}

% Title slide
\title{Digitalisation du Processus de Planification des Rencontres Financières}
\subtitle{Application Web pour la BMCE Capital Investors Conference 2025}
\author{Tolotra David Fernando RAZAFIMAHEFA}
\institute{Institut Supérieur de Management (ISMAGI)}
\date{Juillet 2025}

\begin{document}

% Slide 1: Title
\begin{frame}
\maketitle
\end{frame}

% Slide 2: Remerciements
\begin{frame}{Remerciements}
\begin{itemize}
    \item Gratitude envers ISMAGI pour l'environnement d'apprentissage stimulant
    \item Remerciements à l'encadrante pédagogique, Mme Bouchra HONNIT
    \item Remerciements à l'encadrant professionnel, M. Amine BOUCHAMA, et l'équipe de Wave Inc.
    \item Soutien de la famille et des amis pour leur encouragement constant
\end{itemize}
\end{frame}

% Slide 3: Résumé
\begin{frame}{Résumé}
\begin{itemize}
    \item \textbf{Objectif} : Digitalisation de la planification des réunions pour la BMCE Capital Investors Conference 2025
    \item \textbf{Solution} : Plateforme web avec architecture MVC (Laravel 12, Tailwind CSS, Alpine.js)
    \item \textbf{Fonctionnalités} :
    \begin{itemize}
        \item Gestion des utilisateurs (administrateurs, émetteurs, investisseurs)
        \item Planification automatisée des réunions
        \item Module de questions/réponses
        \item Outils de reporting
    \end{itemize}
    \item \textbf{Bénéfices} : Élimination des erreurs manuelles, autonomie des participants, données analytiques
\end{itemize}
\end{frame}

% Slide 4: Contexte Général
\begin{frame}{Contexte Général}
\begin{itemize}
    \item \textbf{Organisme d'accueil} : Wave Inc., agence digitale à Rabat, spécialisée en transformation numérique
    \item \textbf{Problématique actuelle} :
    \begin{itemize}
        \item Processus manuel chronophage
        \item Risque d'erreurs humaines
        \item Manque de flexibilité et d'autonomie
    \end{itemize}
    \item \textbf{Solution proposée} : Plateforme web centralisée pour une gestion automatisée
\end{itemize}
\end{frame}

% Slide 5: Objectifs
\begin{frame}{Objectifs}
\begin{itemize}
    \item \textbf{Stratégiques} :
    \begin{itemize}
        \item Moderniser l'image de BMCE
        \item Améliorer la satisfaction des participants
        \item Optimiser les ressources humaines
    \end{itemize}
    \item \textbf.Opérationnels} :
    \begin{itemize}
        \item Réduire le temps de planification
        \item Éliminer les erreurs manuelles
        \item Faciliter la gestion des modifications
    \end{itemize}
\end{itemize}
\end{frame}

% Slide 6: Analyse des Besoins
\begin{frame}{Analyse des Besoins}
\begin{itemize}
    \item \textbf{Besoins fonctionnels} :
    \begin{itemize}
        \item Gestion des utilisateurs et authentification
        \item Gestion des disponibilités et réunions
        \item Module de questions/réponses
        \item Génération de plannings et QR codes
        \item Tableaux de bord et reporting
    \end{itemize}
    \item \textbf{Besoins non fonctionnels} :
    \begin{itemize}
        \item Performance : Gestion de 500 utilisateurs simultanés
        \item Sécurité : Chiffrement HTTPS, protection CSRF/XSS
        \item Convivialité : Interface responsive et intuitive
    \end{itemize}
\end{itemize}
\end{frame}

% Slide 7: Conception UML
\begin{frame}{Conception UML}
\begin{itemize}
    \item \textbf{Modélisation} :
    \begin{itemize}
        \item Cas d'utilisation : Gestion des utilisateurs, réunions, disponibilités
        \item Diagrammes de classes : Structure des entités (utilisateurs, réunions, créneaux)
        \item Diagrammes de séquences : Processus de création et réservation
        \item Diagrammes de paquetages : Architecture Laravel
    \end{itemize}
    \item \textbf{Règles de gestion} :
    \begin{itemize}
        \item Rôles : Admin, Émetteur, Investisseur
        \item Créneaux : 45 min, non partageables
        \item Réunions : Statuts (PENDING, CONFIRMED, etc.)
    \end{itemize}
\end{itemize}
\end{frame}

% Slide 8: Technologies Utilisées
\begin{frame}{Technologies \& Outils}
\begin{table}
\begin{tabular}{l l l}
\toprule
\textbf{Nom} & \textbf{Version} & \textbf{Description} \\
\midrule
Laravel & 12.0 & Framework MVC principal \\
Livewire & 3.4 & Interfaces dynamiques en PHP \\
Tailwind CSS & 4.0 & Stylage moderne et responsive \\
MySQL & 8.0 & Base de données relationnelle \\
Vite & 6.2.4 & Build tool pour assets front-end \\
\bottomrule
\end{tabular}
\end{table}
\begin{itemize}
    \item \textbf{Outils} : VS Code, Git, Docker, Laravel Sail
    \item \textbf{Architecture} : MVC pour modularité et maintenabilité
\end{itemize}
\end{frame}

% Slide 9: Réalisation
\begin{frame}{Réalisation du Système}
\begin{itemize}
    \item \textbf{Configuration} :
    \begin{itemize}
        \item Installation de Laravel via Composer
        \item Environnement Docker avec Sail
        \item Base de données MySQL avec migrations
    \end{itemize}
    \item \textbf{Développement} :
    \begin{itemize}
        \item Modèles Eloquent pour relations
        \item Contrôleurs pour logique métier
        \item Vues Blade/Livewire pour interfaces
    \end{itemize}
    \item \textbf{Interfaces} :
    \begin{itemize}
        \item Tableaux de bord (Admin, Émetteur, Investisseur)
        \item Gestion des réunions (tableau, grille, calendrier)
        \item QR codes pour présence
    \end{itemize}
\end{itemize}
\end{frame}

% Slide 10: Interfaces Utilisateur
\begin{frame}{Interfaces Utilisateur}
\begin{itemize}
    \item \textbf{Administrateur} :
    \begin{itemize}
        \item Gestion des utilisateurs, réunions, disponibilités
        \item Statistiques et export de données
    \end{itemize}
    \item \textbf{Émetteur} :
    \begin{itemize}
        \item Gestion des créneaux et réponses aux questions
        \item Tableau de bord avec KPI
    \end{itemize}
    \item \textbf{Investisseur} :
    \begin{itemize}
        \item Demande de réunions, consultation des émetteurs
        \item QR code personnel
    \end{itemize}
\end{itemize}
\end{frame}

% Slide 11: Conclusion
\begin{frame}{Conclusion}
\begin{itemize}
    \item \textbf{Réussites} :
    \begin{itemize}
        \item Digitalisation complète du processus
        \item Interface intuitive et sécurisée
        \item Architecture évolutive avec Laravel
    \end{itemize}
    \item \textbf{Impact} :
    \begin{itemize}
        \item Réduction des erreurs manuelles
        \item Autonomie accrue des participants
        \item Données analytiques pour optimisation
    \end{itemize}
\end{itemize}
\end{frame}

% Slide 12: Perspectives
\begin{frame}{Perspectives}
\begin{itemize}
    \item Support multi-événements
    \item Intégration avec calendriers externes (Google, Outlook)
    \item Messagerie instantanée entre participants
    \item Intelligence artificielle pour suggestions de rencontres
    \item Réunions virtuelles pour format hybride
\end{itemize}
\end{frame}

% Slide 13: Questions
\begin{frame}{Questions}
\begin{center}
\Huge{Questions ?}
\end{center}
\end{frame}

\end{document}
